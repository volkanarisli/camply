The flash is a special area of the session used for storing messages. Messages are written to the flash and cleared after being displayed to the user. The flash is typically used in combination with redirects, ensuring that the message is available to the next page that is to be rendered.

This middleware was extracted from \href{http://expressjs.com/}{\tt Express} 2.\+x, after Express 3.\+x removed direct support for the flash. connect-\/flash brings this functionality back to Express 3.\+x, as well as any other middleware-\/compatible framework or application. +1 for \href{http://substack.net/posts/b96642/the-node-js-aesthetic}{\tt radical reusability}.

\subsection*{Install}

\begin{DoxyVerb}$ npm install connect-flash
\end{DoxyVerb}


\subsection*{Usage}

\paragraph*{Express 3.\+x}

Flash messages are stored in the session. First, setup sessions as usual by enabling {\ttfamily cookie\+Parser} and {\ttfamily session} middleware. Then, use {\ttfamily flash} middleware provided by connect-\/flash.


\begin{DoxyCode}
var flash = require('connect-flash');
var app = express();

app.configure(function() \{
  app.use(express.cookieParser('keyboard cat'));
  app.use(express.session(\{ cookie: \{ maxAge: 60000 \}\}));
  app.use(flash());
\});
\end{DoxyCode}


With the {\ttfamily flash} middleware in place, all requests will have a {\ttfamily req.\+flash()} function that can be used for flash messages.


\begin{DoxyCode}
app.get('/flash', function(req, res)\{
  // Set a flash message by passing the key, followed by the value, to req.flash().
  req.flash('info', 'Flash is back!')
  res.redirect('/');
\});

app.get('/', function(req, res)\{
  // Get an array of flash messages by passing the key to req.flash()
  res.render('index', \{ messages: req.flash('info') \});
\});
\end{DoxyCode}


\subsection*{Examples}

For an example using connect-\/flash in an Express 3.\+x app, refer to the \href{https://github.com/jaredhanson/connect-flash/tree/master/examples/express3}{\tt express3} example.

\subsection*{Tests}

\begin{DoxyVerb}$ npm install --dev
$ make test
\end{DoxyVerb}


\href{http://travis-ci.org/jaredhanson/connect-flash}{\tt }

\subsection*{Credits}


\begin{DoxyItemize}
\item \href{http://github.com/jaredhanson}{\tt Jared Hanson}
\item \href{https://github.com/visionmedia}{\tt TJ Holowaychuk}
\end{DoxyItemize}

\subsection*{License}

\href{http://opensource.org/licenses/MIT}{\tt The M\+IT License}

Copyright (c) 2012-\/2013 Jared Hanson $<$\href{http://jaredhanson.net/}{\tt http\+://jaredhanson.\+net/}$>$ 