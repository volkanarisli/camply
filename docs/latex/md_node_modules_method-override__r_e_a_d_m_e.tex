\href{https://npmjs.org/package/method-override}{\tt } \href{https://npmjs.org/package/method-override}{\tt } \href{https://travis-ci.org/expressjs/method-override}{\tt } \href{https://coveralls.io/r/expressjs/method-override?branch=master}{\tt }

Lets you use H\+T\+TP verbs such as P\+UT or D\+E\+L\+E\+TE in places where the client doesn\textquotesingle{}t support it.

\subsection*{Install}

This is a \href{https://nodejs.org/en/}{\tt Node.\+js} module available through the \href{https://www.npmjs.com/}{\tt npm registry}. Installation is done using the \href{https://docs.npmjs.com/getting-started/installing-npm-packages-locally}{\tt {\ttfamily npm install} command}\+:


\begin{DoxyCode}
$ npm install method-override
\end{DoxyCode}


\subsection*{A\+PI}

{\bfseries N\+O\+TE} It is very important that this module is used {\bfseries before} any module that needs to know the method of the request (for example, it {\itshape must} be used prior to the {\ttfamily csurf} module).

\subsubsection*{method\+Override(getter, options)}

Create a new middleware function to override the {\ttfamily req.\+method} property with a new value. This value will be pulled from the provided {\ttfamily getter}.


\begin{DoxyItemize}
\item {\ttfamily getter} -\/ The getter to use to look up the overridden request method for the request. (default\+: {\ttfamily X-\/\+H\+T\+T\+P-\/\+Method-\/\+Override})
\item {\ttfamily options.\+methods} -\/ The allowed methods the original request must be in to check for a method override value. (default\+: `\mbox{[}\textquotesingle{}P\+O\+ST'\mbox{]}\`{})
\end{DoxyItemize}

If the found method is supported by node.\+js core, then {\ttfamily req.\+method} will be set to this value, as if it has originally been that value. The previous {\ttfamily req.\+method} value will be stored in {\ttfamily req.\+original\+Method}.

\paragraph*{getter}

This is the method of getting the override value from the request. If a function is provided, the {\ttfamily req} is passed as the first argument, the {\ttfamily res} as the second argument and the method is expected to be returned. If a string is provided, the string is used to look up the method with the following rules\+:


\begin{DoxyItemize}
\item If the string starts with {\ttfamily X-\/}, then it is treated as the name of a header and that header is used for the method override. If the request contains the same header multiple times, the first occurrence is used.
\item All other strings are treated as a key in the U\+RL query string.
\end{DoxyItemize}

\paragraph*{options.\+methods}

This allows the specification of what methods(s) the request {\itshape M\+U\+ST} be in in order to check for the method override value. This defaults to only {\ttfamily P\+O\+ST} methods, which is the only method the override should arrive in. More methods may be specified here, but it may introduce security issues and cause weird behavior when requests travel through caches. This value is an array of methods in upper-\/case. {\ttfamily null} can be specified to allow all methods.

\subsection*{Examples}

\subsubsection*{override using a header}

To use a header to override the method, specify the header name as a string argument to the {\ttfamily method\+Override} function. To then make the call, send a {\ttfamily P\+O\+ST} request to a U\+RL with the overridden method as the value of that header. This method of using a header would typically be used in conjunction with {\ttfamily X\+M\+L\+Http\+Request} on implementations that do not support the method you are trying to use.


\begin{DoxyCode}
var express = require('express')
var methodOverride = require('method-override')
var app = express()

// override with the X-HTTP-Method-Override header in the request
app.use(methodOverride('X-HTTP-Method-Override'))
\end{DoxyCode}


Example call with header override using {\ttfamily X\+M\+L\+Http\+Request}\+:


\begin{DoxyCode}
var xhr = new XMLHttpRequest()
xhr.onload = onload
xhr.open('post', '/resource', true)
xhr.setRequestHeader('X-HTTP-Method-Override', 'DELETE')
xhr.send()

function onload () \{
  alert('got response: ' + this.responseText)
\}
\end{DoxyCode}


\subsubsection*{override using a query value}

To use a query string value to override the method, specify the query string key as a string argument to the {\ttfamily method\+Override} function. To then make the call, send a {\ttfamily P\+O\+ST} request to a U\+RL with the overridden method as the value of that query string key. This method of using a query value would typically be used in conjunction with plain H\+T\+ML {\ttfamily $<$form$>$} elements when trying to support legacy browsers but still use newer methods.


\begin{DoxyCode}
var express = require('express')
var methodOverride = require('method-override')
var app = express()

// override with POST having ?\_method=DELETE
app.use(methodOverride('\_method'))
\end{DoxyCode}


Example call with query override using H\+T\+ML {\ttfamily $<$form$>$}\+:


\begin{DoxyCode}
<form method="POST" action="/resource?\_method=DELETE">
  <button type="submit">Delete resource</button>
</form>
\end{DoxyCode}


\subsubsection*{multiple format support}


\begin{DoxyCode}
var express = require('express')
var methodOverride = require('method-override')
var app = express()

// override with different headers; last one takes precedence
app.use(methodOverride('X-HTTP-Method')) //          Microsoft
app.use(methodOverride('X-HTTP-Method-Override')) // Google/GData
app.use(methodOverride('X-Method-Override')) //      IBM
\end{DoxyCode}


\subsubsection*{custom logic}

You can implement any kind of custom logic with a function for the {\ttfamily getter}. The following implements the logic for looking in {\ttfamily req.\+body} that was in {\ttfamily method-\/override@1}\+:


\begin{DoxyCode}
var bodyParser = require('body-parser')
var express = require('express')
var methodOverride = require('method-override')
var app = express()

// NOTE: when using req.body, you must fully parse the request body
//       before you call methodOverride() in your middleware stack,
//       otherwise req.body will not be populated.
app.use(bodyParser.urlencoded())
app.use(methodOverride(function (req, res) \{
  if (req.body && typeof req.body === 'object' && '\_method' in req.body) \{
    // look in urlencoded POST bodies and delete it
    var method = req.body.\_method
    delete req.body.\_method
    return method
  \}
\}))
\end{DoxyCode}


Example call with query override using H\+T\+ML {\ttfamily $<$form$>$}\+:


\begin{DoxyCode}
<form method="POST" action="/resource" enctype="application/x-www-form-urlencoded">
  <input type="hidden" name="\_method" value="DELETE">
  <button type="submit">Delete resource</button>
</form>
\end{DoxyCode}


\subsection*{License}

\mbox{[}M\+IT\mbox{]}(L\+I\+C\+E\+N\+SE) 