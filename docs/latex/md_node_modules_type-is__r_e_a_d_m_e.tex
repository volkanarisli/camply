\href{https://npmjs.org/package/type-is}{\tt } \href{https://npmjs.org/package/type-is}{\tt } \href{https://nodejs.org/en/download/}{\tt } \href{https://travis-ci.org/jshttp/type-is}{\tt } \href{https://coveralls.io/r/jshttp/type-is?branch=master}{\tt }

Infer the content-\/type of a request.

\subsubsection*{Install}

This is a \href{https://nodejs.org/en/}{\tt Node.\+js} module available through the \href{https://www.npmjs.com/}{\tt npm registry}. Installation is done using the \href{https://docs.npmjs.com/getting-started/installing-npm-packages-locally}{\tt {\ttfamily npm install} command}\+:


\begin{DoxyCode}
$ npm install type-is
\end{DoxyCode}


\subsection*{A\+PI}


\begin{DoxyCode}
var http = require('http')
var typeis = require('type-is')

http.createServer(function (req, res) \{
  var istext = typeis(req, ['text/*'])
  res.end('you ' + (istext ? 'sent' : 'did not send') + ' me text')
\})
\end{DoxyCode}


\subsubsection*{type = typeis(request, types)}

{\ttfamily request} is the node H\+T\+TP request. {\ttfamily types} is an array of types.


\begin{DoxyCode}
// req.headers.content-type = 'application/json'

typeis(req, ['json'])             // 'json'
typeis(req, ['html', 'json'])     // 'json'
typeis(req, ['application/*'])    // 'application/json'
typeis(req, ['application/json']) // 'application/json'

typeis(req, ['html']) // false
\end{DoxyCode}


\subsubsection*{typeis.\+has\+Body(request)}

Returns a Boolean if the given {\ttfamily request} has a body, regardless of the {\ttfamily Content-\/\+Type} header.

Having a body has no relation to how large the body is (it may be 0 bytes). This is similar to how file existence works. If a body does exist, then this indicates that there is data to read from the Node.\+js request stream.


\begin{DoxyCode}
if (typeis.hasBody(req)) \{
  // read the body, since there is one

  req.on('data', function (chunk) \{
    // ...
  \})
\}
\end{DoxyCode}


\subsubsection*{type = typeis.\+is(media\+Type, types)}

{\ttfamily media\+Type} is the \href{https://tools.ietf.org/html/rfc6838}{\tt media type} string. {\ttfamily types} is an array of types.


\begin{DoxyCode}
var mediaType = 'application/json'

typeis.is(mediaType, ['json'])             // 'json'
typeis.is(mediaType, ['html', 'json'])     // 'json'
typeis.is(mediaType, ['application/*'])    // 'application/json'
typeis.is(mediaType, ['application/json']) // 'application/json'

typeis.is(mediaType, ['html']) // false
\end{DoxyCode}


\subsubsection*{Each type can be\+:}


\begin{DoxyItemize}
\item An extension name such as {\ttfamily json}. This name will be returned if matched.
\item A mime type such as {\ttfamily application/json}.
\item A mime type with a wildcard such as {\ttfamily $\ast$/$\ast$} or {\ttfamily $\ast$/json} or {\ttfamily application/$\ast$}. The full mime type will be returned if matched.
\item A suffix such as {\ttfamily +json}. This can be combined with a wildcard such as {\ttfamily $\ast$/vnd+json} or {\ttfamily application/$\ast$+json}. The full mime type will be returned if matched.
\end{DoxyItemize}

{\ttfamily false} will be returned if no type matches or the content type is invalid.

{\ttfamily null} will be returned if the request does not have a body.

\subsection*{Examples}

\subsubsection*{Example body parser}


\begin{DoxyCode}
var express = require('express')
var typeis = require('type-is')

var app = express()

app.use(function bodyParser (req, res, next) \{
  if (!typeis.hasBody(req)) \{
    return next()
  \}

  switch (typeis(req, ['urlencoded', 'json', 'multipart'])) \{
    case 'urlencoded':
      // parse urlencoded body
      throw new Error('implement urlencoded body parsing')
    case 'json':
      // parse json body
      throw new Error('implement json body parsing')
    case 'multipart':
      // parse multipart body
      throw new Error('implement multipart body parsing')
    default:
      // 415 error code
      res.statusCode = 415
      res.end()
      break
  \}
\})
\end{DoxyCode}


\subsection*{License}

\mbox{[}M\+IT\mbox{]}(L\+I\+C\+E\+N\+SE) 