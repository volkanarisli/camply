\href{https://npmjs.org/package/mime-db}{\tt } \href{https://npmjs.org/package/mime-db}{\tt } \href{https://nodejs.org/en/download}{\tt } \href{https://travis-ci.org/jshttp/mime-db}{\tt } \href{https://coveralls.io/r/jshttp/mime-db?branch=master}{\tt }

This is a database of all mime types. It consists of a single, public J\+S\+ON file and does not include any logic, allowing it to remain as un-\/opinionated as possible with an A\+PI. It aggregates data from the following sources\+:


\begin{DoxyItemize}
\item \href{http://www.iana.org/assignments/media-types/media-types.xhtml}{\tt http\+://www.\+iana.\+org/assignments/media-\/types/media-\/types.\+xhtml}
\item \href{http://svn.apache.org/repos/asf/httpd/httpd/trunk/docs/conf/mime.types}{\tt http\+://svn.\+apache.\+org/repos/asf/httpd/httpd/trunk/docs/conf/mime.\+types}
\item \href{http://hg.nginx.org/nginx/raw-file/default/conf/mime.types}{\tt http\+://hg.\+nginx.\+org/nginx/raw-\/file/default/conf/mime.\+types}
\end{DoxyItemize}

\subsection*{Installation}


\begin{DoxyCode}
npm install mime-db
\end{DoxyCode}


\subsubsection*{Database Download}

If you\textquotesingle{}re crazy enough to use this in the browser, you can just grab the J\+S\+ON file using \href{https://rawgit.com/}{\tt Raw\+Git}. It is recommended to replace {\ttfamily master} with \href{https://github.com/jshttp/mime-db/tags}{\tt a release tag} as the J\+S\+ON format may change in the future.


\begin{DoxyCode}
https://cdn.rawgit.com/jshttp/mime-db/master/db.json
\end{DoxyCode}


\subsection*{Usage}


\begin{DoxyCode}
var db = require('mime-db');

// grab data on .js files
var data = db['application/javascript'];
\end{DoxyCode}


\subsection*{Data Structure}

The J\+S\+ON file is a map lookup for lowercased mime types. Each mime type has the following properties\+:


\begin{DoxyItemize}
\item {\ttfamily .source} -\/ where the mime type is defined. If not set, it\textquotesingle{}s probably a custom media type.
\begin{DoxyItemize}
\item {\ttfamily apache} -\/ \href{http://svn.apache.org/repos/asf/httpd/httpd/trunk/docs/conf/mime.types}{\tt Apache common media types}
\item {\ttfamily iana} -\/ \href{http://www.iana.org/assignments/media-types/media-types.xhtml}{\tt I\+A\+N\+A-\/defined media types}
\item {\ttfamily nginx} -\/ \href{http://hg.nginx.org/nginx/raw-file/default/conf/mime.types}{\tt nginx media types}
\end{DoxyItemize}
\item {\ttfamily .extensions\mbox{[}\mbox{]}} -\/ known extensions associated with this mime type.
\item {\ttfamily .compressible} -\/ whether a file of this type can be gzipped.
\item {\ttfamily .charset} -\/ the default charset associated with this type, if any.
\end{DoxyItemize}

If unknown, every property could be {\ttfamily undefined}.

\subsection*{Contributing}

To edit the database, only make P\+Rs against {\ttfamily src/custom.\+json} or {\ttfamily src/custom-\/suffix.\+json}.

The {\ttfamily src/custom.\+json} file is a J\+S\+ON object with the M\+I\+ME type as the keys and the values being an object with the following keys\+:


\begin{DoxyItemize}
\item {\ttfamily compressible} -\/ leave out if you don\textquotesingle{}t know, otherwise {\ttfamily true}/{\ttfamily false} to indicate whether the data represented by the type is typically compressible.
\item {\ttfamily extensions} -\/ include an array of file extensions that are associated with the type.
\item {\ttfamily notes} -\/ human-\/readable notes about the type, typically what the type is.
\item {\ttfamily sources} -\/ include an array of U\+R\+Ls of where the M\+I\+ME type and the associated extensions are sourced from. This needs to be a \href{https://en.wikipedia.org/wiki/Primary_source}{\tt primary source}; links to type aggregating sites and Wikipedia are {\itshape not acceptable}.
\end{DoxyItemize}

To update the build, run {\ttfamily npm run build}.

\subsection*{Adding Custom Media Types}

The best way to get new media types included in this library is to register them with the I\+A\+NA. The community registration procedure is outlined in \href{http://tools.ietf.org/html/rfc6838#section-5}{\tt R\+FC 6838 section 5}. Types registered with the I\+A\+NA are automatically pulled into this library. 