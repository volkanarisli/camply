\href{https://npmjs.org/package/serve-static}{\tt } \href{https://npmjs.org/package/serve-static}{\tt } \href{https://travis-ci.org/expressjs/serve-static}{\tt } \href{https://ci.appveyor.com/project/dougwilson/serve-static}{\tt } \href{https://coveralls.io/r/expressjs/serve-static}{\tt }

\subsection*{Install}

This is a \href{https://nodejs.org/en/}{\tt Node.\+js} module available through the \href{https://www.npmjs.com/}{\tt npm registry}. Installation is done using the \href{https://docs.npmjs.com/getting-started/installing-npm-packages-locally}{\tt {\ttfamily npm install} command}\+:


\begin{DoxyCode}
$ npm install serve-static
\end{DoxyCode}


\subsection*{A\+PI}


\begin{DoxyCode}
var serveStatic = require('serve-static')
\end{DoxyCode}


\subsubsection*{serve\+Static(root, options)}

Create a new middleware function to serve files from within a given root directory. The file to serve will be determined by combining {\ttfamily req.\+url} with the provided root directory. When a file is not found, instead of sending a 404 response, this module will instead call {\ttfamily next()} to move on to the next middleware, allowing for stacking and fall-\/backs.

\paragraph*{Options}

\subparagraph*{accept\+Ranges}

Enable or disable accepting ranged requests, defaults to true. Disabling this will not send {\ttfamily Accept-\/\+Ranges} and ignore the contents of the {\ttfamily Range} request header.

\subparagraph*{cache\+Control}

Enable or disable setting {\ttfamily Cache-\/\+Control} response header, defaults to true. Disabling this will ignore the {\ttfamily immutable} and {\ttfamily max\+Age} options.

\subparagraph*{dotfiles}

Set how \char`\"{}dotfiles\char`\"{} are treated when encountered. A dotfile is a file or directory that begins with a dot (\char`\"{}.\char`\"{}). Note this check is done on the path itself without checking if the path actually exists on the disk. If {\ttfamily root} is specified, only the dotfiles above the root are checked (i.\+e. the root itself can be within a dotfile when set to \char`\"{}deny\char`\"{}).


\begin{DoxyItemize}
\item `\textquotesingle{}allow'{\ttfamily No special treatment for dotfiles. -\/}\textquotesingle{}deny\textquotesingle{}{\ttfamily Deny a request for a dotfile and 403/}next(){\ttfamily . -\/}\textquotesingle{}ignore\textquotesingle{}{\ttfamily Pretend like the dotfile does not exist and 404/}next()\`{}.
\end{DoxyItemize}

The default value is similar to `\textquotesingle{}ignore'\`{}, with the exception that this default will not ignore the files within a directory that begins with a dot.

\subparagraph*{etag}

Enable or disable etag generation, defaults to true.

\subparagraph*{extensions}

Set file extension fallbacks. When set, if a file is not found, the given extensions will be added to the file name and search for. The first that exists will be served. Example\+: `\mbox{[}\textquotesingle{}html', \textquotesingle{}htm\textquotesingle{}\mbox{]}\`{}.

The default value is {\ttfamily false}.

\subparagraph*{fallthrough}

Set the middleware to have client errors fall-\/through as just unhandled requests, otherwise forward a client error. The difference is that client errors like a bad request or a request to a non-\/existent file will cause this middleware to simply {\ttfamily next()} to your next middleware when this value is {\ttfamily true}. When this value is {\ttfamily false}, these errors (even 404s), will invoke {\ttfamily next(err)}.

Typically {\ttfamily true} is desired such that multiple physical directories can be mapped to the same web address or for routes to fill in non-\/existent files.

The value {\ttfamily false} can be used if this middleware is mounted at a path that is designed to be strictly a single file system directory, which allows for short-\/circuiting 404s for less overhead. This middleware will also reply to all methods.

The default value is {\ttfamily true}.

\subparagraph*{immutable}

Enable or disable the {\ttfamily immutable} directive in the {\ttfamily Cache-\/\+Control} response header, defaults to {\ttfamily false}. If set to {\ttfamily true}, the {\ttfamily max\+Age} option should also be specified to enable caching. The {\ttfamily immutable} directive will prevent supported clients from making conditional requests during the life of the {\ttfamily max\+Age} option to check if the file has changed.

\subparagraph*{index}

By default this module will send \char`\"{}index.\+html\char`\"{} files in response to a request on a directory. To disable this set {\ttfamily false} or to supply a new index pass a string or an array in preferred order.

\subparagraph*{last\+Modified}

Enable or disable {\ttfamily Last-\/\+Modified} header, defaults to true. Uses the file system\textquotesingle{}s last modified value.

\subparagraph*{max\+Age}

Provide a max-\/age in milliseconds for http caching, defaults to 0. This can also be a string accepted by the \href{https://www.npmjs.org/package/ms#readme}{\tt ms} module.

\subparagraph*{redirect}

Redirect to trailing \char`\"{}/\char`\"{} when the pathname is a dir. Defaults to {\ttfamily true}.

\subparagraph*{set\+Headers}

Function to set custom headers on response. Alterations to the headers need to occur synchronously. The function is called as {\ttfamily fn(res, path, stat)}, where the arguments are\+:


\begin{DoxyItemize}
\item {\ttfamily res} the response object
\item {\ttfamily path} the file path that is being sent
\item {\ttfamily stat} the stat object of the file that is being sent
\end{DoxyItemize}

\subsection*{Examples}

\subsubsection*{Serve files with vanilla node.\+js http server}


\begin{DoxyCode}
var finalhandler = require('finalhandler')
var http = require('http')
var serveStatic = require('serve-static')

// Serve up public/ftp folder
var serve = serveStatic('public/ftp', \{'index': ['index.html', 'index.htm']\})

// Create server
var server = http.createServer(function onRequest (req, res) \{
  serve(req, res, finalhandler(req, res))
\})

// Listen
server.listen(3000)
\end{DoxyCode}


\subsubsection*{Serve all files as downloads}


\begin{DoxyCode}
var contentDisposition = require('content-disposition')
var finalhandler = require('finalhandler')
var http = require('http')
var serveStatic = require('serve-static')

// Serve up public/ftp folder
var serve = serveStatic('public/ftp', \{
  'index': false,
  'setHeaders': setHeaders
\})

// Set header to force download
function setHeaders (res, path) \{
  res.setHeader('Content-Disposition', contentDisposition(path))
\}

// Create server
var server = http.createServer(function onRequest (req, res) \{
  serve(req, res, finalhandler(req, res))
\})

// Listen
server.listen(3000)
\end{DoxyCode}


\subsubsection*{Serving using express}

\paragraph*{Simple}

This is a simple example of using Express.


\begin{DoxyCode}
var express = require('express')
var serveStatic = require('serve-static')

var app = express()

app.use(serveStatic('public/ftp', \{'index': ['default.html', 'default.htm']\}))
app.listen(3000)
\end{DoxyCode}


\paragraph*{Multiple roots}

This example shows a simple way to search through multiple directories. Files are look for in {\ttfamily public-\/optimized/} first, then {\ttfamily public/} second as a fallback.


\begin{DoxyCode}
var express = require('express')
var path = require('path')
var serveStatic = require('serve-static')

var app = express()

app.use(serveStatic(path.join(\_\_dirname, 'public-optimized')))
app.use(serveStatic(path.join(\_\_dirname, 'public')))
app.listen(3000)
\end{DoxyCode}


\paragraph*{Different settings for paths}

This example shows how to set a different max age depending on the served file type. In this example, H\+T\+ML files are not cached, while everything else is for 1 day.


\begin{DoxyCode}
var express = require('express')
var path = require('path')
var serveStatic = require('serve-static')

var app = express()

app.use(serveStatic(path.join(\_\_dirname, 'public'), \{
  maxAge: '1d',
  setHeaders: setCustomCacheControl
\}))

app.listen(3000)

function setCustomCacheControl (res, path) \{
  if (serveStatic.mime.lookup(path) === 'text/html') \{
    // Custom Cache-Control for HTML files
    res.setHeader('Cache-Control', 'public, max-age=0')
  \}
\}
\end{DoxyCode}


\subsection*{License}

\mbox{[}M\+IT\mbox{]}(L\+I\+C\+E\+N\+SE) 