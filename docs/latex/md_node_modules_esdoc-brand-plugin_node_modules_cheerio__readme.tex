\section*{cheerio}

\subparagraph*{Fast, flexible \& lean implementation of core j\+Query designed specifically for the server.}

 \href{http://travis-ci.org/cheeriojs/cheerio}{\tt } \href{https://coveralls.io/r/cheeriojs/cheerio}{\tt } \href{https://gitter.im/cheeriojs/cheerio?utm_source=badge&utm_medium=badge&utm_campaign=pr-badge&utm_content=badge}{\tt } \href{#backers}{\tt } \href{#sponsors}{\tt } 

~\newline



\begin{DoxyCode}
let cheerio = require('cheerio')
let $ = cheerio.load('<h2 class="title">Hello world</h2>')

$('h2.title').text('Hello there!')
$('h2').addClass('welcome')

$.html()
//=> <h2 class="title welcome">Hello there!</h2>
\end{DoxyCode}


\subsection*{Installation}

{\ttfamily npm install cheerio}

\subsection*{Features}

{\bfseries \&\#10084; Familiar syntax\+:} Cheerio implements a subset of core j\+Query. Cheerio removes all the D\+OM inconsistencies and browser cruft from the j\+Query library, revealing its truly gorgeous A\+PI.

{\bfseries \&\#991; Blazingly fast\+:} Cheerio works with a very simple, consistent D\+OM model. As a result parsing, manipulating, and rendering are incredibly efficient. Preliminary end-\/to-\/end benchmarks suggest that cheerio is about {\bfseries 8x} faster than J\+S\+D\+OM.

{\bfseries \&\#10049; Incredibly flexible\+:} Cheerio wraps around \textquotesingle{}s forgiving \href{https://github.com/fb55/htmlparser2/}{\tt htmlparser2}. Cheerio can parse nearly any H\+T\+ML or X\+ML document.

\subsection*{Sponsors}

Does your company use Cheerio in production? Please consider \href{https://opencollective.com/cheerio#sponsor}{\tt sponsoring this project}. Your help will allow maintainers to dedicate more time and resources to its development and support.

\href{https://opencollective.com/cheerio/sponsor/0/website}{\tt } \href{https://opencollective.com/cheerio/sponsor/1/website}{\tt } \href{https://opencollective.com/cheerio/sponsor/2/website}{\tt } \href{https://opencollective.com/cheerio/sponsor/3/website}{\tt } \href{https://opencollective.com/cheerio/sponsor/4/website}{\tt } \href{https://opencollective.com/cheerio/sponsor/5/website}{\tt } \href{https://opencollective.com/cheerio/sponsor/6/website}{\tt } \href{https://opencollective.com/cheerio/sponsor/7/website}{\tt } \href{https://opencollective.com/cheerio/sponsor/8/website}{\tt } \href{https://opencollective.com/cheerio/sponsor/9/website}{\tt } \href{https://opencollective.com/cheerio/sponsor/10/website}{\tt } \href{https://opencollective.com/cheerio/sponsor/11/website}{\tt } \href{https://opencollective.com/cheerio/sponsor/12/website}{\tt } \href{https://opencollective.com/cheerio/sponsor/13/website}{\tt } \href{https://opencollective.com/cheerio/sponsor/14/website}{\tt } \href{https://opencollective.com/cheerio/sponsor/15/website}{\tt } \href{https://opencollective.com/cheerio/sponsor/16/website}{\tt } \href{https://opencollective.com/cheerio/sponsor/17/website}{\tt } \href{https://opencollective.com/cheerio/sponsor/18/website}{\tt } \href{https://opencollective.com/cheerio/sponsor/19/website}{\tt } \href{https://opencollective.com/cheerio/sponsor/20/website}{\tt } \href{https://opencollective.com/cheerio/sponsor/21/website}{\tt } \href{https://opencollective.com/cheerio/sponsor/22/website}{\tt } \href{https://opencollective.com/cheerio/sponsor/23/website}{\tt } \href{https://opencollective.com/cheerio/sponsor/24/website}{\tt } \href{https://opencollective.com/cheerio/sponsor/25/website}{\tt } \href{https://opencollective.com/cheerio/sponsor/26/website}{\tt } \href{https://opencollective.com/cheerio/sponsor/27/website}{\tt } \href{https://opencollective.com/cheerio/sponsor/28/website}{\tt } \href{https://opencollective.com/cheerio/sponsor/29/website}{\tt }

\subsection*{Backers}

\href{https://opencollective.com/cheerio#backer}{\tt Become a backer} to show your support for Cheerio and help us maintain and improve this open source project.

\href{https://opencollective.com/cheerio/backer/0/website}{\tt } \href{https://opencollective.com/cheerio/backer/1/website}{\tt } \href{https://opencollective.com/cheerio/backer/2/website}{\tt } \href{https://opencollective.com/cheerio/backer/3/website}{\tt } \href{https://opencollective.com/cheerio/backer/4/website}{\tt } \href{https://opencollective.com/cheerio/backer/5/website}{\tt } \href{https://opencollective.com/cheerio/backer/6/website}{\tt } \href{https://opencollective.com/cheerio/backer/7/website}{\tt } \href{https://opencollective.com/cheerio/backer/8/website}{\tt } \href{https://opencollective.com/cheerio/backer/9/website}{\tt } \href{https://opencollective.com/cheerio/backer/10/website}{\tt } \href{https://opencollective.com/cheerio/backer/11/website}{\tt } \href{https://opencollective.com/cheerio/backer/12/website}{\tt } \href{https://opencollective.com/cheerio/backer/13/website}{\tt } \href{https://opencollective.com/cheerio/backer/14/website}{\tt } \href{https://opencollective.com/cheerio/backer/15/website}{\tt } \href{https://opencollective.com/cheerio/backer/16/website}{\tt } \href{https://opencollective.com/cheerio/backer/17/website}{\tt } \href{https://opencollective.com/cheerio/backer/18/website}{\tt } \href{https://opencollective.com/cheerio/backer/19/website}{\tt } \href{https://opencollective.com/cheerio/backer/20/website}{\tt } \href{https://opencollective.com/cheerio/backer/21/website}{\tt } \href{https://opencollective.com/cheerio/backer/22/website}{\tt } \href{https://opencollective.com/cheerio/backer/23/website}{\tt } \href{https://opencollective.com/cheerio/backer/24/website}{\tt } \href{https://opencollective.com/cheerio/backer/25/website}{\tt } \href{https://opencollective.com/cheerio/backer/26/website}{\tt } \href{https://opencollective.com/cheerio/backer/27/website}{\tt } \href{https://opencollective.com/cheerio/backer/28/website}{\tt } \href{https://opencollective.com/cheerio/backer/29/website}{\tt }

\subsection*{A\+PI}

\subsubsection*{Markup example we\textquotesingle{}ll be using\+:}


\begin{DoxyCode}
<ul id="fruits">
  <li class="apple">Apple</li>
  <li class="orange">Orange</li>
  <li class="pear">Pear</li>
</ul>
\end{DoxyCode}


This is the H\+T\+ML markup we will be using in all of the A\+PI examples.

\subsubsection*{Loading}

First you need to load in the H\+T\+ML. This step in j\+Query is implicit, since j\+Query operates on the one, baked-\/in D\+OM. With Cheerio, we need to pass in the H\+T\+ML document.

This is the {\itshape preferred} method\+:


\begin{DoxyCode}
var cheerio = require('cheerio'),
    $ = cheerio.load('<ul id="fruits">...</ul>');
\end{DoxyCode}


Optionally, you can also load in the H\+T\+ML by passing the string as the context\+:


\begin{DoxyCode}
$ = require('cheerio');
$('ul', '<ul id="fruits">...</ul>');
\end{DoxyCode}


Or as the root\+:


\begin{DoxyCode}
$ = require('cheerio');
$('li', 'ul', '<ul id="fruits">...</ul>');
\end{DoxyCode}


You can also pass an extra object to {\ttfamily .load()} if you need to modify any of the default parsing options\+:


\begin{DoxyCode}
$ = cheerio.load('<ul id="fruits">...</ul>', \{
    normalizeWhitespace: true,
    xmlMode: true
\});
\end{DoxyCode}


These parsing options are taken directly from \href{https://github.com/fb55/htmlparser2/wiki/Parser-options}{\tt htmlparser2}, therefore any options that can be used in {\ttfamily htmlparser2} are valid in cheerio as well. The default options are\+:


\begin{DoxyCode}
\{
    withDomLvl1: true,
    normalizeWhitespace: false,
    xmlMode: false,
    decodeEntities: true
\}
\end{DoxyCode}


For a full list of options and their effects, see \href{https://github.com/fb55/DomHandler}{\tt this} and \href{https://github.com/fb55/htmlparser2/wiki/Parser-options}{\tt htmlparser2\textquotesingle{}s options}.

\subsubsection*{Selectors}

Cheerio\textquotesingle{}s selector implementation is nearly identical to j\+Query\textquotesingle{}s, so the A\+PI is very similar.

\paragraph*{\$( selector, \mbox{[}context\mbox{]}, \mbox{[}root\mbox{]} )}

{\ttfamily selector} searches within the {\ttfamily context} scope which searches within the {\ttfamily root} scope. {\ttfamily selector} and {\ttfamily context} can be a string expression, D\+OM Element, array of D\+OM elements, or cheerio object. {\ttfamily root} is typically the H\+T\+ML document string.

This selector method is the starting point for traversing and manipulating the document. Like j\+Query, it\textquotesingle{}s the primary method for selecting elements in the document, but unlike j\+Query it\textquotesingle{}s built on top of the C\+S\+S\+Select library, which implements most of the Sizzle selectors.


\begin{DoxyCode}
$('.apple', '#fruits').text()
//=> Apple

$('ul .pear').attr('class')
//=> pear

$('li[class=orange]').html()
//=> Orange
\end{DoxyCode}


\subsubsection*{Attributes}

Methods for getting and modifying attributes.

\paragraph*{.attr( name, value )}

Method for getting and setting attributes. Gets the attribute value for only the first element in the matched set. If you set an attribute\textquotesingle{}s value to {\ttfamily null}, you remove that attribute. You may also pass a {\ttfamily map} and {\ttfamily function} like j\+Query.


\begin{DoxyCode}
$('ul').attr('id')
//=> fruits

$('.apple').attr('id', 'favorite').html()
//=> <li class="apple" id="favorite">Apple</li>
\end{DoxyCode}


\begin{quote}
See \href{http://api.jquery.com/attr/}{\tt http\+://api.\+jquery.\+com/attr/} for more information \end{quote}


\paragraph*{.prop( name, value )}

Method for getting and setting properties. Gets the property value for only the first element in the matched set.


\begin{DoxyCode}
$('input[type="checkbox"]').prop('checked')
//=> false

$('input[type="checkbox"]').prop('checked', true).val()
//=> ok
\end{DoxyCode}


\begin{quote}
See \href{http://api.jquery.com/prop/}{\tt http\+://api.\+jquery.\+com/prop/} for more information \end{quote}


\paragraph*{.data( name, value )}

Method for getting and setting data attributes. Gets or sets the data attribute value for only the first element in the matched set.


\begin{DoxyCode}
$('<div data-apple-color="red"></div>').data()
//=> \{ appleColor: 'red' \}

$('<div data-apple-color="red"></div>').data('apple-color')
//=> 'red'

var apple = $('.apple').data('kind', 'mac')
apple.data('kind')
//=> 'mac'
\end{DoxyCode}


\begin{quote}
See \href{http://api.jquery.com/data/}{\tt http\+://api.\+jquery.\+com/data/} for more information \end{quote}


\paragraph*{.val( \mbox{[}value\mbox{]} )}

Method for getting and setting the value of input, select, and textarea. Note\+: Support for {\ttfamily map}, and {\ttfamily function} has not been added yet.


\begin{DoxyCode}
$('input[type="text"]').val()
//=> input\_text

$('input[type="text"]').val('test').html()
//=> <input type="text" value="test"/>
\end{DoxyCode}


\paragraph*{.remove\+Attr( name )}

Method for removing attributes by {\ttfamily name}.


\begin{DoxyCode}
$('.pear').removeAttr('class').html()
//=> <li>Pear</li>
\end{DoxyCode}


\paragraph*{.has\+Class( class\+Name )}

Check to see if {\itshape any} of the matched elements have the given {\ttfamily class\+Name}.


\begin{DoxyCode}
$('.pear').hasClass('pear')
//=> true

$('apple').hasClass('fruit')
//=> false

$('li').hasClass('pear')
//=> true
\end{DoxyCode}


\paragraph*{.add\+Class( class\+Name )}

Adds class(es) to all of the matched elements. Also accepts a {\ttfamily function} like j\+Query.


\begin{DoxyCode}
$('.pear').addClass('fruit').html()
//=> <li class="pear fruit">Pear</li>

$('.apple').addClass('fruit red').html()
//=> <li class="apple fruit red">Apple</li>
\end{DoxyCode}


\begin{quote}
See \href{http://api.jquery.com/addClass/}{\tt http\+://api.\+jquery.\+com/add\+Class/} for more information. \end{quote}


\paragraph*{.remove\+Class( \mbox{[}class\+Name\mbox{]} )}

Removes one or more space-\/separated classes from the selected elements. If no {\ttfamily class\+Name} is defined, all classes will be removed. Also accepts a {\ttfamily function} like j\+Query.


\begin{DoxyCode}
$('.pear').removeClass('pear').html()
//=> <li class="">Pear</li>

$('.apple').addClass('red').removeClass().html()
//=> <li class="">Apple</li>
\end{DoxyCode}


\begin{quote}
See \href{http://api.jquery.com/removeClass/}{\tt http\+://api.\+jquery.\+com/remove\+Class/} for more information. \end{quote}


\paragraph*{.toggle\+Class( class\+Name, \mbox{[}switch\mbox{]} )}

Add or remove class(es) from the matched elements, depending on either the class\textquotesingle{}s presence or the value of the switch argument. Also accepts a {\ttfamily function} like j\+Query.


\begin{DoxyCode}
$('.apple.green').toggleClass('fruit green red').html()
//=> <li class="apple fruit red">Apple</li>

$('.apple.green').toggleClass('fruit green red', true).html()
//=> <li class="apple green fruit red">Apple</li>
\end{DoxyCode}


\begin{quote}
See \href{http://api.jquery.com/toggleClass/}{\tt http\+://api.\+jquery.\+com/toggle\+Class/} for more information. \end{quote}


\paragraph*{.is( selector )}

\paragraph*{.is( element )}

\paragraph*{.is( selection )}

\paragraph*{.is( function(index) )}

Checks the current list of elements and returns {\ttfamily true} if {\itshape any} of the elements match the selector. If using an element or Cheerio selection, returns {\ttfamily true} if {\itshape any} of the elements match. If using a predicate function, the function is executed in the context of the selected element, so {\ttfamily this} refers to the current element.

\subsubsection*{Forms}

\paragraph*{.serialize\+Array()}

Encode a set of form elements as an array of names and values.


\begin{DoxyCode}
$('<form><input name="foo" value="bar" /></form>').serializeArray()
//=> [ \{ name: 'foo', value: 'bar' \} ]
\end{DoxyCode}


\subsubsection*{Traversing}

\paragraph*{.find(selector)}

\paragraph*{.find(selection)}

\paragraph*{.find(node)}

Get the descendants of each element in the current set of matched elements, filtered by a selector, j\+Query object, or element.


\begin{DoxyCode}
$('#fruits').find('li').length
//=> 3
$('#fruits').find($('.apple')).length
//=> 1
\end{DoxyCode}


\paragraph*{.parent(\mbox{[}selector\mbox{]})}

Get the parent of each element in the current set of matched elements, optionally filtered by a selector.


\begin{DoxyCode}
$('.pear').parent().attr('id')
//=> fruits
\end{DoxyCode}


\paragraph*{.parents(\mbox{[}selector\mbox{]})}

Get a set of parents filtered by {\ttfamily selector} of each element in the current set of match elements. 
\begin{DoxyCode}
$('.orange').parents().length
// => 2
$('.orange').parents('#fruits').length
// => 1
\end{DoxyCode}


\paragraph*{.parents\+Until(\mbox{[}selector\mbox{]}\mbox{[},filter\mbox{]})}

Get the ancestors of each element in the current set of matched elements, up to but not including the element matched by the selector, D\+OM node, or cheerio object. 
\begin{DoxyCode}
$('.orange').parentsUntil('#food').length
// => 1
\end{DoxyCode}


\paragraph*{.closest(selector)}

For each element in the set, get the first element that matches the selector by testing the element itself and traversing up through its ancestors in the D\+OM tree.


\begin{DoxyCode}
$('.orange').closest()
// => []
$('.orange').closest('.apple')
// => []
$('.orange').closest('li')
// => [<li class="orange">Orange</li>]
$('.orange').closest('#fruits')
// => [<ul id="fruits"> ... </ul>]
\end{DoxyCode}


\paragraph*{.next(\mbox{[}selector\mbox{]})}

Gets the next sibling of the first selected element, optionally filtered by a selector.


\begin{DoxyCode}
$('.apple').next().hasClass('orange')
//=> true
\end{DoxyCode}


\paragraph*{.next\+All(\mbox{[}selector\mbox{]})}

Gets all the following siblings of the first selected element, optionally filtered by a selector.


\begin{DoxyCode}
$('.apple').nextAll()
//=> [<li class="orange">Orange</li>, <li class="pear">Pear</li>]
$('.apple').nextAll('.orange')
//=> [<li class="orange">Orange</li>]
\end{DoxyCode}


\paragraph*{.next\+Until(\mbox{[}selector\mbox{]}, \mbox{[}filter\mbox{]})}

Gets all the following siblings up to but not including the element matched by the selector, optionally filtered by another selector.


\begin{DoxyCode}
$('.apple').nextUntil('.pear')
//=> [<li class="orange">Orange</li>]
\end{DoxyCode}


\paragraph*{.prev(\mbox{[}selector\mbox{]})}

Gets the previous sibling of the first selected element optionally filtered by a selector.


\begin{DoxyCode}
$('.orange').prev().hasClass('apple')
//=> true
\end{DoxyCode}


\paragraph*{.prev\+All(\mbox{[}selector\mbox{]})}

Gets all the preceding siblings of the first selected element, optionally filtered by a selector.


\begin{DoxyCode}
$('.pear').prevAll()
//=> [<li class="orange">Orange</li>, <li class="apple">Apple</li>]
$('.pear').prevAll('.orange')
//=> [<li class="orange">Orange</li>]
\end{DoxyCode}


\paragraph*{.prev\+Until(\mbox{[}selector\mbox{]}, \mbox{[}filter\mbox{]})}

Gets all the preceding siblings up to but not including the element matched by the selector, optionally filtered by another selector.


\begin{DoxyCode}
$('.pear').prevUntil('.apple')
//=> [<li class="orange">Orange</li>]
\end{DoxyCode}


\paragraph*{.slice( start, \mbox{[}end\mbox{]} )}

Gets the elements matching the specified range


\begin{DoxyCode}
$('li').slice(1).eq(0).text()
//=> 'Orange'

$('li').slice(1, 2).length
//=> 1
\end{DoxyCode}


\paragraph*{.siblings(\mbox{[}selector\mbox{]})}

Gets the first selected element\textquotesingle{}s siblings, excluding itself.


\begin{DoxyCode}
$('.pear').siblings().length
//=> 2

$('.pear').siblings('.orange').length
//=> 1
\end{DoxyCode}


\paragraph*{.children(\mbox{[}selector\mbox{]})}

Gets the children of the first selected element.


\begin{DoxyCode}
$('#fruits').children().length
//=> 3

$('#fruits').children('.pear').text()
//=> Pear
\end{DoxyCode}


\paragraph*{.contents()}

Gets the children of each element in the set of matched elements, including text and comment nodes.


\begin{DoxyCode}
$('#fruits').contents().length
//=> 3
\end{DoxyCode}


\paragraph*{.each( function(index, element) )}

Iterates over a cheerio object, executing a function for each matched element. When the callback is fired, the function is fired in the context of the D\+OM element, so {\ttfamily this} refers to the current element, which is equivalent to the function parameter {\ttfamily element}. To break out of the {\ttfamily each} loop early, return with {\ttfamily false}.


\begin{DoxyCode}
var fruits = [];

$('li').each(function(i, elem) \{
  fruits[i] = $(this).text();
\});

fruits.join(', ');
//=> Apple, Orange, Pear
\end{DoxyCode}


\paragraph*{.map( function(index, element) )}

Pass each element in the current matched set through a function, producing a new Cheerio object containing the return values. The function can return an individual data item or an array of data items to be inserted into the resulting set. If an array is returned, the elements inside the array are inserted into the set. If the function returns null or undefined, no element will be inserted.


\begin{DoxyCode}
$('li').map(function(i, el) \{
  // this === el
  return $(this).text();
\}).get().join(' ');
//=> "apple orange pear"
\end{DoxyCode}


\paragraph*{.filter( selector ) ~\newline
 .filter( selection ) ~\newline
 .filter( element ) ~\newline
 .filter( function(index) )}

Iterates over a cheerio object, reducing the set of selector elements to those that match the selector or pass the function\textquotesingle{}s test. When a Cheerio selection is specified, return only the elements contained in that selection. When an element is specified, return only that element (if it is contained in the original selection). If using the function method, the function is executed in the context of the selected element, so {\ttfamily this} refers to the current element.

Selector\+:


\begin{DoxyCode}
$('li').filter('.orange').attr('class');
//=> orange
\end{DoxyCode}


Function\+:


\begin{DoxyCode}
$('li').filter(function(i, el) \{
  // this === el
  return $(this).attr('class') === 'orange';
\}).attr('class')
//=> orange
\end{DoxyCode}


\paragraph*{.not( selector ) ~\newline
 .not( selection ) ~\newline
 .not( element ) ~\newline
 .not( function(index, elem) )}

Remove elements from the set of matched elements. Given a j\+Query object that represents a set of D\+OM elements, the {\ttfamily .not()} method constructs a new j\+Query object from a subset of the matching elements. The supplied selector is tested against each element; the elements that don\textquotesingle{}t match the selector will be included in the result. The {\ttfamily .not()} method can take a function as its argument in the same way that {\ttfamily .filter()} does. Elements for which the function returns true are excluded from the filtered set; all other elements are included.

Selector\+:


\begin{DoxyCode}
$('li').not('.apple').length;
//=> 2
\end{DoxyCode}


Function\+:


\begin{DoxyCode}
$('li').not(function(i, el) \{
  // this === el
  return $(this).attr('class') === 'orange';
\}).length;
//=> 2
\end{DoxyCode}


\paragraph*{.has( selector ) ~\newline
 .has( element )}

Filters the set of matched elements to only those which have the given D\+OM element as a descendant or which have a descendant that matches the given selector. Equivalent to `.filter('\+:has(selector)\textquotesingle{})\`{}.

Selector\+:


\begin{DoxyCode}
$('ul').has('.pear').attr('id');
//=> fruits
\end{DoxyCode}


Element\+:


\begin{DoxyCode}
$('ul').has($('.pear')[0]).attr('id');
//=> fruits
\end{DoxyCode}


\paragraph*{.first()}

Will select the first element of a cheerio object


\begin{DoxyCode}
$('#fruits').children().first().text()
//=> Apple
\end{DoxyCode}


\paragraph*{.last()}

Will select the last element of a cheerio object


\begin{DoxyCode}
$('#fruits').children().last().text()
//=> Pear
\end{DoxyCode}


\paragraph*{.eq( i )}

Reduce the set of matched elements to the one at the specified index. Use {\ttfamily .eq(-\/i)} to count backwards from the last selected element.


\begin{DoxyCode}
$('li').eq(0).text()
//=> Apple

$('li').eq(-1).text()
//=> Pear
\end{DoxyCode}


\paragraph*{.get( \mbox{[}i\mbox{]} )}

Retrieve the D\+OM elements matched by the Cheerio object. If an index is specified, retrieve one of the elements matched by the Cheerio object\+:


\begin{DoxyCode}
$('li').get(0).tagName
//=> li
\end{DoxyCode}


If no index is specified, retrieve all elements matched by the Cheerio object\+:


\begin{DoxyCode}
$('li').get().length
//=> 3
\end{DoxyCode}


\paragraph*{.index()}

\paragraph*{.index( selector )}

\paragraph*{.index( node\+Or\+Selection )}

Search for a given element from among the matched elements.


\begin{DoxyCode}
$('.pear').index()
//=> 2
$('.orange').index('li')
//=> 1
$('.apple').index($('#fruit, li'))
//=> 1
\end{DoxyCode}


\paragraph*{.end()}

End the most recent filtering operation in the current chain and return the set of matched elements to its previous state.


\begin{DoxyCode}
$('li').eq(0).end().length
//=> 3
\end{DoxyCode}


\paragraph*{.add( selector \mbox{[}, context\mbox{]} )}

\paragraph*{.add( element )}

\paragraph*{.add( elements )}

\paragraph*{.add( html )}

\paragraph*{.add( selection )}

Add elements to the set of matched elements.


\begin{DoxyCode}
$('.apple').add('.orange').length
//=> 2
\end{DoxyCode}


\paragraph*{.add\+Back( \mbox{[}filter\mbox{]} )}

Add the previous set of elements on the stack to the current set, optionally filtered by a selector.


\begin{DoxyCode}
$('li').eq(0).addBack('.orange').length
//=> 2
\end{DoxyCode}


\subsubsection*{Manipulation}

Methods for modifying the D\+OM structure.

\paragraph*{.append( content, \mbox{[}content, ...\mbox{]} )}

Inserts content as the {\itshape last} child of each of the selected elements.


\begin{DoxyCode}
$('ul').append('<li class="plum">Plum</li>')
$.html()
//=>  <ul id="fruits">
//      <li class="apple">Apple</li>
//      <li class="orange">Orange</li>
//      <li class="pear">Pear</li>
//      <li class="plum">Plum</li>
//    </ul>
\end{DoxyCode}


\paragraph*{.append\+To( target )}

Insert every element in the set of matched elements to the end of the target.


\begin{DoxyCode}
$('<li class="plum">Plum</li>').appendTo('#fruits')
$.html()
//=>  <ul id="fruits">
//      <li class="apple">Apple</li>
//      <li class="orange">Orange</li>
//      <li class="pear">Pear</li>
//      <li class="plum">Plum</li>
//    </ul>
\end{DoxyCode}


\paragraph*{.prepend( content, \mbox{[}content, ...\mbox{]} )}

Inserts content as the {\itshape first} child of each of the selected elements.


\begin{DoxyCode}
$('ul').prepend('<li class="plum">Plum</li>')
$.html()
//=>  <ul id="fruits">
//      <li class="plum">Plum</li>
//      <li class="apple">Apple</li>
//      <li class="orange">Orange</li>
//      <li class="pear">Pear</li>
//    </ul>
\end{DoxyCode}


\paragraph*{.prepend\+To( target )}

Insert every element in the set of matched elements to the beginning of the target.


\begin{DoxyCode}
$('<li class="plum">Plum</li>').prependTo('#fruits')
$.html()
//=>  <ul id="fruits">
//      <li class="plum">Plum</li>
//      <li class="apple">Apple</li>
//      <li class="orange">Orange</li>
//      <li class="pear">Pear</li>
//    </ul>
\end{DoxyCode}


\paragraph*{.after( content, \mbox{[}content, ...\mbox{]} )}

Insert content next to each element in the set of matched elements.


\begin{DoxyCode}
$('.apple').after('<li class="plum">Plum</li>')
$.html()
//=>  <ul id="fruits">
//      <li class="apple">Apple</li>
//      <li class="plum">Plum</li>
//      <li class="orange">Orange</li>
//      <li class="pear">Pear</li>
//    </ul>
\end{DoxyCode}


\paragraph*{.insert\+After( target )}

Insert every element in the set of matched elements after the target.


\begin{DoxyCode}
$('<li class="plum">Plum</li>').insertAfter('.apple')
$.html()
//=>  <ul id="fruits">
//      <li class="apple">Apple</li>
//      <li class="plum">Plum</li>
//      <li class="orange">Orange</li>
//      <li class="pear">Pear</li>
//    </ul>
\end{DoxyCode}


\paragraph*{.before( content, \mbox{[}content, ...\mbox{]} )}

Insert content previous to each element in the set of matched elements.


\begin{DoxyCode}
$('.apple').before('<li class="plum">Plum</li>')
$.html()
//=>  <ul id="fruits">
//      <li class="plum">Plum</li>
//      <li class="apple">Apple</li>
//      <li class="orange">Orange</li>
//      <li class="pear">Pear</li>
//    </ul>
\end{DoxyCode}


\paragraph*{.insert\+Before( target )}

Insert every element in the set of matched elements before the target.


\begin{DoxyCode}
$('<li class="plum">Plum</li>').insertBefore('.apple')
$.html()
//=>  <ul id="fruits">
//      <li class="plum">Plum</li>
//      <li class="apple">Apple</li>
//      <li class="orange">Orange</li>
//      <li class="pear">Pear</li>
//    </ul>
\end{DoxyCode}


\paragraph*{.remove( \mbox{[}selector\mbox{]} )}

Removes the set of matched elements from the D\+OM and all their children. {\ttfamily selector} filters the set of matched elements to be removed.


\begin{DoxyCode}
$('.pear').remove()
$.html()
//=>  <ul id="fruits">
//      <li class="apple">Apple</li>
//      <li class="orange">Orange</li>
//    </ul>
\end{DoxyCode}


\paragraph*{.replace\+With( content )}

Replaces matched elements with {\ttfamily content}.


\begin{DoxyCode}
var plum = $('<li class="plum">Plum</li>')
$('.pear').replaceWith(plum)
$.html()
//=> <ul id="fruits">
//     <li class="apple">Apple</li>
//     <li class="orange">Orange</li>
//     <li class="plum">Plum</li>
//   </ul>
\end{DoxyCode}


\paragraph*{.empty()}

Empties an element, removing all its children.


\begin{DoxyCode}
$('ul').empty()
$.html()
//=>  <ul id="fruits"></ul>
\end{DoxyCode}


\paragraph*{.html( \mbox{[}html\+String\mbox{]} )}

Gets an html content string from the first selected element. If {\ttfamily html\+String} is specified, each selected element\textquotesingle{}s content is replaced by the new content.


\begin{DoxyCode}
$('.orange').html()
//=> Orange

$('#fruits').html('<li class="mango">Mango</li>').html()
//=> <li class="mango">Mango</li>
\end{DoxyCode}


\paragraph*{.text( \mbox{[}text\+String\mbox{]} )}

Get the combined text contents of each element in the set of matched elements, including their descendants.. If {\ttfamily text\+String} is specified, each selected element\textquotesingle{}s content is replaced by the new text content.


\begin{DoxyCode}
$('.orange').text()
//=> Orange

$('ul').text()
//=>  Apple
//    Orange
//    Pear
\end{DoxyCode}


\paragraph*{.wrap( content )}

The .wrap() function can take any string or object that could be passed to the \$() factory function to specify a D\+OM structure. This structure may be nested several levels deep, but should contain only one inmost element. A copy of this structure will be wrapped around each of the elements in the set of matched elements. This method returns the original set of elements for chaining purposes.


\begin{DoxyCode}
var redFruit = $('<div class="red-fruit"></div>')
$('.apple').wrap(redFruit)

//=> <ul id="fruits">
//     <div class="red-fruit">
//      <li class="apple">Apple</li>
//     </div>
//     <li class="orange">Orange</li>
//     <li class="plum">Plum</li>
//   </ul>

var healthy = $('<div class="healthy"></div>')
$('li').wrap(healthy)

//=> <ul id="fruits">
//     <div class="healthy">
//       <li class="apple">Apple</li>
//     </div>
//     <div class="healthy">
//       <li class="orange">Orange</li>
//     </div>
//     <div class="healthy">
//        <li class="plum">Plum</li>
//     </div>
//   </ul>
\end{DoxyCode}


\paragraph*{.css( \mbox{[}propert\+Name\mbox{]} ) ~\newline
 .css( \mbox{[} property\+Names\mbox{]} ) ~\newline
 .css( \mbox{[}property\+Name\mbox{]}, \mbox{[}value\mbox{]} ) ~\newline
 .css( \mbox{[}propert\+Name\mbox{]}, \mbox{[}function\mbox{]} ) ~\newline
 .css( \mbox{[}properties\mbox{]} )}

Get the value of a style property for the first element in the set of matched elements or set one or more C\+SS properties for every matched element.

\subsubsection*{Rendering}

When you\textquotesingle{}re ready to render the document, you can use the {\ttfamily html} utility function\+:


\begin{DoxyCode}
$.html()
//=>  <ul id="fruits">
//      <li class="apple">Apple</li>
//      <li class="orange">Orange</li>
//      <li class="pear">Pear</li>
//    </ul>
\end{DoxyCode}


If you want to return the outer\+H\+T\+ML you can use {\ttfamily \$.html(selector)}\+:


\begin{DoxyCode}
$.html('.pear')
//=> <li class="pear">Pear</li>
\end{DoxyCode}


By default, {\ttfamily html} will leave some tags open. Sometimes you may instead want to render a valid X\+ML document. For example, you might parse the following X\+ML snippet\+:


\begin{DoxyCode}
$ = \textcolor{keyword}{cheerio.load}(\textcolor{stringliteral}{'<media:thumbnail url="http://www.foo.com/keyframe.jpg" width="75" height="50"
       time="12:05:01.123"/>'});
\end{DoxyCode}


... and later want to render to X\+ML. To do this, you can use the \textquotesingle{}xml\textquotesingle{} utility function\+:


\begin{DoxyCode}
$.xml()
//=>  <media:thumbnail url="http://www.foo.com/keyframe.jpg" width="75" height="50" time="12:05:01.123"/>
\end{DoxyCode}


You may also render the text content of a Cheerio object using the {\ttfamily text} static method\+:


\begin{DoxyCode}
$ = cheerio.load('This is <em>content</em>.')
$.text()
//=> This is content.
\end{DoxyCode}


The method may be called on the Cheerio module itself--be sure to pass a collection of nodes!


\begin{DoxyCode}
$ = cheerio.load('<div>This is <em>content</em>.</div>')
cheerio.text($('div'))
//=> This is content.
\end{DoxyCode}


\subsubsection*{Miscellaneous}

D\+OM element methods that don\textquotesingle{}t fit anywhere else

\paragraph*{.to\+Array()}

Retrieve all the D\+OM elements contained in the j\+Query set as an array.


\begin{DoxyCode}
$('li').toArray()
//=> [ \{...\}, \{...\}, \{...\} ]
\end{DoxyCode}


\paragraph*{.clone()}

Clone the cheerio object.


\begin{DoxyCode}
var moreFruit = $('#fruits').clone()
\end{DoxyCode}


\subsubsection*{Utilities}

\paragraph*{\$.root}

Sometimes you need to work with the top-\/level root element. To query it, you can use {\ttfamily \$.root()}.


\begin{DoxyCode}
$.root().append('<ul id="vegetables"></ul>').html();
//=> <ul id="fruits">...</ul><ul id="vegetables"></ul>
\end{DoxyCode}


\paragraph*{\$.contains( container, contained )}

Checks to see if the {\ttfamily contained} D\+OM element is a descendant of the {\ttfamily container} D\+OM element.

\paragraph*{\$.parse\+H\+T\+M\+L( data \mbox{[}, context \mbox{]} \mbox{[}, keep\+Scripts \mbox{]} )}

Parses a string into an array of D\+OM nodes. The {\ttfamily context} argument has no meaning for Cheerio, but it is maintained for A\+PI compatability.

\paragraph*{\$.load( html\mbox{[}, options \mbox{]} )}

Load in the H\+T\+ML. (See the previous section titled \char`\"{}\+Loading\char`\"{} for more information.)

\subsubsection*{Plugins}

Once you have loaded a document, you may extend the prototype or the equivalent {\ttfamily fn} property with custom plugin methods\+:


\begin{DoxyCode}
var $ = cheerio.load('<html><body>Hello, <b>world</b>!</body></html>');
$.prototype.logHtml = function() \{
  console.log(this.html());
\};

$('body').logHtml(); // logs "Hello, <b>world</b>!" to the console
\end{DoxyCode}


\subsubsection*{The \char`\"{}\+D\+O\+M Node\char`\"{} object}

Cheerio collections are made up of objects that bear some resemblence to \href{https://developer.mozilla.org/en-US/docs/Web/API/Node}{\tt browser-\/based D\+OM nodes}. You can expect them to define the following properties\+:


\begin{DoxyItemize}
\item {\ttfamily tag\+Name}
\item {\ttfamily parent\+Node}
\item {\ttfamily previous\+Sibling}
\item {\ttfamily next\+Sibling}
\item {\ttfamily node\+Value}
\item {\ttfamily first\+Child}
\item {\ttfamily child\+Nodes}
\item {\ttfamily last\+Child}
\end{DoxyItemize}

\subsection*{What about J\+S\+D\+OM?}

I wrote cheerio because I found myself increasingly frustrated with J\+S\+D\+OM. For me, there were three main sticking points that I kept running into again and again\+:

{\bfseries \&\#8226; J\+S\+D\+OM\textquotesingle{}s built-\/in parser is too strict\+:} J\+S\+D\+OM\textquotesingle{}s bundled H\+T\+ML parser cannot handle many popular sites out there today.

{\bfseries \&\#8226; J\+S\+D\+OM is too slow\+:} Parsing big websites with J\+S\+D\+OM has a noticeable delay.

{\bfseries \&\#8226; J\+S\+D\+OM feels too heavy\+:} The goal of J\+S\+D\+OM is to provide an identical D\+OM environment as what we see in the browser. I never really needed all this, I just wanted a simple, familiar way to do H\+T\+ML manipulation.

\subsection*{When I would use J\+S\+D\+OM}

Cheerio will not solve all your problems. I would still use J\+S\+D\+OM if I needed to work in a browser-\/like environment on the server, particularly if I wanted to automate functional tests.

\subsection*{Screencasts}

\href{http://vimeo.com/31950192}{\tt http\+://vimeo.\+com/31950192}

\begin{quote}
This video tutorial is a follow-\/up to Nettut\textquotesingle{}s \char`\"{}\+How to Scrape Web Pages with Node.\+js and j\+Query\char`\"{}, using cheerio instead of J\+S\+D\+OM + j\+Query. This video shows how easy it is to use cheerio and how much faster cheerio is than J\+S\+D\+OM + j\+Query. \end{quote}


\subsection*{Contributors}

These are some of the contributors that have made cheerio possible\+:


\begin{DoxyCode}
project  : cheerio
 repo age : 2 years, 6 months
 active   : 285 days
 commits  : 762
 files    : 36
 authors  :
   293  Matt Mueller            38.5%
   133  Matthew Mueller         17.5%
    92  Mike Pennisi            12.1%
    54  David Chambers          7.1%
    30  kpdecker                3.9%
    19  Felix Böhm             2.5%
    17  fb55                    2.2%
    15  Siddharth Mahendraker   2.0%
    11  Adam Bretz              1.4%
     8  Nazar Leush             1.0%
     7  ironchefpython          0.9%
     6  Jarno Leppänen         0.8%
     5  Ben Sheldon             0.7%
     5  Jos Shepherd            0.7%
     5  Ryan Schmukler          0.7%
     5  Steven Vachon           0.7%
     4  Maciej Adwent           0.5%
     4  Amir Abu Shareb         0.5%
     3  jeremy.dentel@brandingbrand.com 0.4%
     3  Andi Neck               0.4%
     2  steve                   0.3%
     2  alexbardas              0.3%
     2  finspin                 0.3%
     2  Ali Farhadi             0.3%
     2  Chris Khoo              0.3%
     2  Rob Ashton              0.3%
     2  Thomas Heymann          0.3%
     2  Jaro Spisak             0.3%
     2  Dan Dascalescu          0.3%
     2  Torstein Thune          0.3%
     2  Wayne Larsen            0.3%
     1  Timm Preetz             0.1%
     1  Xavi                    0.1%
     1  Alex Shaindlin          0.1%
     1  mattym                  0.1%
     1  Felix Böhm            0.1%
     1  Farid Neshat            0.1%
     1  Dmitry Mazuro           0.1%
     1  Jeremy Hubble           0.1%
     1  nevermind               0.1%
     1  Manuel Alabor           0.1%
     1  Matt Liegey             0.1%
     1  Chris O'Hara            0.1%
     1  Michael Holroyd         0.1%
     1  Michiel De Mey          0.1%
     1  Ben Atkin               0.1%
     1  Rich Trott              0.1%
     1  Rob "Hurricane" Ashton  0.1%
     1  Robin Gloster           0.1%
     1  Simon Boudrias          0.1%
     1  Sindre Sorhus           0.1%
     1  xiaohwan                0.1%
\end{DoxyCode}


\subsection*{Cheerio in the real world}

Are you using cheerio in production? Add it to the \href{https://github.com/cheeriojs/cheerio/wiki/Cheerio-in-Production}{\tt wiki}!

\subsection*{Testing}

To run the test suite, download the repository, then within the cheerio directory, run\+:


\begin{DoxyCode}
make setup
make test
\end{DoxyCode}


This will download the development packages and run the test suite.

\subsection*{Special Thanks}

This library stands on the shoulders of some incredible developers. A special thanks to\+:

\+\_\+\+\_\+\&\#8226;  for node-\/htmlparser2 \& C\+S\+S\+Select\+:\+\_\+\+\_\+ Felix has a knack for writing speedy parsing engines. He completely re-\/wrote both \textquotesingle{}s {\ttfamily node-\/htmlparser} and \textquotesingle{}s {\ttfamily node-\/soupselect} from the ground up, making both of them much faster and more flexible. Cheerio would not be possible without his foundational work

\+\_\+\+\_\+\&\#8226;  team for j\+Query\+:\+\_\+\+\_\+ The core A\+PI is the best of its class and despite dealing with all the browser inconsistencies the code base is extremely clean and easy to follow. Much of cheerio\textquotesingle{}s implementation and documentation is from j\+Query. Thanks guys.

\+\_\+\+\_\+\&\#8226; \+:\+\_\+\+\_\+ The style, the structure, the open-\/source\char`\"{}-\/ness\char`\"{} of this library comes from studying TJ\textquotesingle{}s style and using many of his libraries. This dude consistently pumps out high-\/quality libraries and has always been more than willing to help or answer questions. You rock TJ.

\subsection*{License}

M\+IT 