 

\href{https://travis-ci.org/inikulin/parse5}{\tt } \href{https://www.npmjs.com/package/parse5}{\tt }

{\itshape W\+H\+A\+T\+WG H\+T\+M\+L5 specification-\/compliant, fast and ready for production H\+T\+ML parsing/serialization toolset for Node and io.\+js.}

I needed fast and ready for production H\+T\+ML parser, which will parse H\+T\+ML as a modern browser\textquotesingle{}s parser. Existing solutions were either too slow or their output was too inaccurate. So, this is how parse5 was born.

{\bfseries Included tools\+:}
\begin{DoxyItemize}
\item \href{#class-parser}{\tt Parser} -\/ H\+T\+ML to D\+O\+M-\/tree parser.
\item \href{#class-simpleapiparser}{\tt Simple\+Api\+Parser} -\/ \href{http://en.wikipedia.org/wiki/Simple_API_for_XML}{\tt S\+AX}-\/style parser for H\+T\+ML.
\item \href{#class-serializer}{\tt Serializer} -\/ D\+O\+M-\/tree to H\+T\+ML code serializer.
\end{DoxyItemize}

\#\# Install 
\begin{DoxyCode}
$ npm install parse5
\end{DoxyCode}


\#\# Usage 
\begin{DoxyCode}
var Parser = require('parse5').Parser;

//Instantiate parser
var parser = new Parser();

//Then feed it with an HTML document
var document = parser.parse('<!DOCTYPE html><html><head></head><body>Hi there!</body></html>')

//Now let's parse HTML-snippet
var fragment = parser.parseFragment('<title>Parse5 is &#102;&#117;&#99;&#107;ing
       awesome!</title><h1>42</h1>');
\end{DoxyCode}


\subsection*{Is it fast?}

Check out \href{https://github.com/inikulin/node-html-parser-bench}{\tt this benchmark}.


\begin{DoxyCode}
Starting benchmark. Fasten your seatbelts...
html5 (https://github.com/aredridel/html5) x 0.18 ops/sec ±5.92% (5 runs sampled)
htmlparser (https://github.com/tautologistics/node-htmlparser/) x 3.83 ops/sec ±42.43% (14 runs sampled)
htmlparser2 (https://github.com/fb55/htmlparser2) x 4.05 ops/sec ±39.27% (15 runs sampled)
parse5 (https://github.com/inikulin/parse5) x 3.04 ops/sec ±51.81% (13 runs sampled)
Fastest is htmlparser2 (https://github.com/fb55/htmlparser2),parse5 (https://github.com/inikulin/parse5)
\end{DoxyCode}


So, parse5 is as fast as simple specification incompatible parsers and $\sim$15-\/times(!) faster than the current specification compatible parser available for the node.

\subsection*{A\+PI reference}

\subsubsection*{Enum\+: Tree\+Adapters}

Provides built-\/in tree adapters which can be passed as an optional argument to the {\ttfamily Parser} and {\ttfamily Serializer} constructors.

\paragraph*{\textbullet{} Tree\+Adapters.\+default}

Default tree format for parse5.

\paragraph*{\textbullet{} Tree\+Adapters.\+htmlparser2}

Quite popular \href{https://github.com/fb55/htmlparser2}{\tt htmlparser2} tree format (e.\+g. used in \href{https://github.com/MatthewMueller/cheerio}{\tt cheerio} and \href{https://github.com/tmpvar/jsdom}{\tt jsdom}). 



\subsubsection*{Class\+: Parser}

Provides H\+T\+ML parsing functionality.

\paragraph*{\textbullet{} Parser.\+ctor(\mbox{[}tree\+Adapter, options\mbox{]})}

Creates new reusable instance of the {\ttfamily Parser}. Optional {\ttfamily tree\+Adapter} argument specifies resulting tree format. If {\ttfamily tree\+Adapter} argument is not specified, {\ttfamily default} tree adapter will be used.

{\ttfamily options} object provides the parsing algorithm modifications\+: \subparagraph*{options.\+decode\+Html\+Entities}

Decode H\+T\+M\+L-\/entities like {\ttfamily \&}, {\ttfamily ~}, etc. Default\+: {\ttfamily true}. {\bfseries Warning\+:} disabling this option may cause output which is not conform H\+T\+M\+L5 specification. \subparagraph*{options.\+location\+Info}

Enables source code location information for the nodes. Default\+: {\ttfamily false}. When enabled, each node (except root node) has {\ttfamily \+\_\+\+\_\+location} property, which contains {\ttfamily start} and {\ttfamily end} indices of the node in the source code. If element was implicitly created by the parser it\textquotesingle{}s {\ttfamily \+\_\+\+\_\+location} property will be {\ttfamily null}. In case the node is not an empty element, {\ttfamily \+\_\+\+\_\+location} has two addition properties {\ttfamily start\+Tag} and {\ttfamily end\+Tag} which contain location information for individual tags in a fashion similar to {\ttfamily \+\_\+\+\_\+location} property.

{\itshape Example\+:} 
\begin{DoxyCode}
var parse5 = require('parse5');

//Instantiate new parser with default tree adapter
var parser1 = new parse5.Parser();

//Instantiate new parser with htmlparser2 tree adapter
var parser2 = new parse5.Parser(parse5.TreeAdapters.htmlparser2);
\end{DoxyCode}


\paragraph*{\textbullet{} Parser.\+parse(html)}

Parses specified {\ttfamily html} string. Returns {\ttfamily document} node.

{\itshape Example\+:} 
\begin{DoxyCode}
var document = parser.parse('<!DOCTYPE html><html><head></head><body>Hi there!</body></html>');
\end{DoxyCode}


\paragraph*{\textbullet{} Parser.\+parse\+Fragment(html\+Fragment, \mbox{[}context\+Element\mbox{]})}

Parses given {\ttfamily html\+Fragment}. Returns {\ttfamily document\+Fragment} node. Optional {\ttfamily context\+Element} argument specifies context in which given {\ttfamily html\+Fragment} will be parsed (consider it as setting {\ttfamily context\+Element.\+inner\+H\+T\+ML} property). If {\ttfamily context\+Element} argument is not specified then {\ttfamily $<$template$>$} element will be used as a context and fragment will be parsed in \textquotesingle{}forgiving\textquotesingle{} manner.

{\itshape Example\+:} 
\begin{DoxyCode}
var documentFragment = parser.parseFragment('<table></table>');

//Parse html fragment in context of the parsed <table> element
var trFragment = parser.parseFragment('<tr><td>Shake it, baby</td></tr>', documentFragment.childNodes[0]);
\end{DoxyCode}
 



\subsubsection*{Class\+: Simple\+Api\+Parser}

Provides \href{https://en.wikipedia.org/wiki/Simple_API_for_XML}{\tt S\+AX}-\/style H\+T\+ML parsing functionality.

\paragraph*{\textbullet{} Simple\+Api\+Parser.\+ctor(handlers, \mbox{[}options\mbox{]})}

Creates new reusable instance of the {\ttfamily Simple\+Api\+Parser}. {\ttfamily handlers} argument specifies object that contains parser\textquotesingle{}s event handlers. Possible events and their signatures are shown in the example.

{\ttfamily options} object provides the parsing algorithm modifications\+: \subparagraph*{options.\+decode\+Html\+Entities}

Decode H\+T\+M\+L-\/entities like {\ttfamily \&}, {\ttfamily ~}, etc. Default\+: {\ttfamily true}. {\bfseries Warning\+:} disabling this option may cause output which is not conform H\+T\+M\+L5 specification. \subparagraph*{options.\+location\+Info}

Enables source code location information for the tokens. Default\+: {\ttfamily false}. When enabled, each node handler receives {\ttfamily location} object as it\textquotesingle{}s last argument. {\ttfamily location} object contains {\ttfamily start} and {\ttfamily end} indices of the token in the source code.

{\itshape Example\+:} 
\begin{DoxyCode}
var parse5 = require('parse5');

var parser = new parse5.SimpleApiParser(\{
    doctype: function(name, publicId, systemId /*, [location] */) \{
        //Handle doctype here
    \},

    startTag: function(tagName, attrs, selfClosing /*, [location] */) \{
        //Handle start tags here
    \},

    endTag: function(tagName /*, [location] */) \{
        //Handle end tags here
    \},

    text: function(text /*, [location] */) \{
        //Handle texts here
    \},

    comment: function(text /*, [location] */) \{
        //Handle comments here
    \}
\});
\end{DoxyCode}


\paragraph*{\textbullet{} Simple\+Api\+Parser.\+parse(html)}

Raises parser events for the given {\ttfamily html}.

{\itshape Example\+:} 
\begin{DoxyCode}
var parse5 = require('parse5');

var parser = new parse5.SimpleApiParser(\{
    text: function(text) \{
        console.log(text);
    \}
\});

parser.parse('<body>Yo!</body>');
\end{DoxyCode}
 



\subsubsection*{Class\+: Serializer}

Provides tree-\/to-\/\+H\+T\+ML serialization functionality. {\bfseries Note\+:} prior to v1.\+2.\+0 this class was called {\ttfamily Tree\+Serializer}. However, it\textquotesingle{}s still accessible as {\ttfamily parse5.\+Tree\+Serializer} for backward compatibility.

\paragraph*{\textbullet{} Serializer.\+ctor(\mbox{[}tree\+Adapter, options\mbox{]})}

Creates new reusable instance of the {\ttfamily Serializer}. Optional {\ttfamily tree\+Adapter} argument specifies input tree format. If {\ttfamily tree\+Adapter} argument is not specified, {\ttfamily default} tree adapter will be used.

{\ttfamily options} object provides the serialization algorithm modifications\+: \subparagraph*{options.\+encode\+Html\+Entities}

H\+T\+M\+L-\/encode characters like {\ttfamily $<$}, {\ttfamily $>$}, {\ttfamily \&}, etc. Default\+: {\ttfamily true}. {\bfseries Warning\+:} disabling this option may cause output which is not conform H\+T\+M\+L5 specification.

{\itshape Example\+:} 
\begin{DoxyCode}
var parse5 = require('parse5');

//Instantiate new serializer with default tree adapter
var serializer1 = new parse5.Serializer();

//Instantiate new serializer with htmlparser2 tree adapter
var serializer2 = new parse5.Serializer(parse5.TreeAdapters.htmlparser2);
\end{DoxyCode}


\paragraph*{\textbullet{} Serializer.\+serialize(node)}

Serializes the given {\ttfamily node}. Returns H\+T\+ML string.

{\itshape Example\+:} 
\begin{DoxyCode}
var document = parser.parse('<!DOCTYPE html><html><head></head><body>Hi there!</body></html>');

//Serialize document
var html = serializer.serialize(document);

//Serialize <body> element content
var bodyInnerHtml = serializer.serialize(document.childNodes[0].childNodes[1]);
\end{DoxyCode}
 



\subsection*{Testing}

Test data is adopted from \href{https://github.com/html5lib}{\tt html5lib project}. Parser is covered by more than 8000 test cases. To run tests\+: 
\begin{DoxyCode}
$ npm test
\end{DoxyCode}


\subsection*{Custom tree adapter}

You can create a custom tree adapter so parse5 can work with your own D\+O\+M-\/tree implementation. Just pass your adapter implementation to the parser\textquotesingle{}s constructor as an argument\+:


\begin{DoxyCode}
var Parser = require('parse5').Parser;

var myTreeAdapter = \{
   //Adapter methods...
\};

//Instantiate parser
var parser = new Parser(myTreeAdapter);
\end{DoxyCode}


Sample implementation can be found \href{https://github.com/inikulin/parse5/blob/master/lib/tree_adapters/default.js}{\tt here}. The custom tree adapter should implement all methods exposed via {\ttfamily exports} in the sample implementation.

\subsection*{Questions or suggestions?}

If you have any questions, please feel free to create an issue \href{https://github.com/inikulin/parse5/issues}{\tt here on github}.

\subsection*{Author}

\href{https://github.com/inikulin}{\tt Ivan Nikulin} (\href{mailto:ifaaan@gmail.com}{\tt ifaaan@gmail.\+com}) 