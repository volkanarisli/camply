\section*{~\newline
 ~\newline
  ~\newline
 ~\newline
 ~\newline
 }

\begin{quote}
Terminal string styling done right \end{quote}


\href{https://travis-ci.org/chalk/chalk}{\tt } \href{https://coveralls.io/github/chalk/chalk?branch=master}{\tt } \href{https://www.youtube.com/watch?v=9auOCbH5Ns4}{\tt } \href{https://github.com/xojs/xo}{\tt } \href{https://github.com/sindresorhus/awesome-nodejs}{\tt }

\subsubsection*{\href{https://github.com/chalk/chalk/releases/tag/v2.0.0}{\tt See what\textquotesingle{}s new in Chalk 2}}



\subsection*{Highlights}


\begin{DoxyItemize}
\item Expressive A\+PI
\item Highly performant
\item Ability to nest styles
\item \href{#256-and-truecolor-color-support}{\tt 256/\+Truecolor color support}
\item Auto-\/detects color support
\item Doesn\textquotesingle{}t extend {\ttfamily String.\+prototype}
\item Clean and focused
\item Actively maintained
\item \href{https://www.npmjs.com/browse/depended/chalk}{\tt Used by $\sim$23,000 packages} as of December 31, 2017
\end{DoxyItemize}

\subsection*{Install}


\begin{DoxyCode}
$ npm install chalk
\end{DoxyCode}


\href{https://www.patreon.com/sindresorhus}{\tt }

\subsection*{Usage}


\begin{DoxyCode}
const chalk = require('chalk');

console.log(chalk.blue('Hello world!'));
\end{DoxyCode}


Chalk comes with an easy to use composable A\+PI where you just chain and nest the styles you want.

\`{}\`{}`js const chalk = require(\textquotesingle{}chalk'); const log = console.\+log;

// Combine styled and normal strings log(chalk.\+blue(\textquotesingle{}Hello\textquotesingle{}) + \textquotesingle{} World\textquotesingle{} + chalk.\+red(\textquotesingle{}!\textquotesingle{}));

// Compose multiple styles using the chainable A\+PI log(chalk.\+blue.\+bg\+Red.\+bold(\textquotesingle{}Hello world!\textquotesingle{}));

// Pass in multiple arguments log(chalk.\+blue(\textquotesingle{}Hello\textquotesingle{}, \textquotesingle{}World!\textquotesingle{}, \textquotesingle{}Foo\textquotesingle{}, \textquotesingle{}bar\textquotesingle{}, \textquotesingle{}biz\textquotesingle{}, \textquotesingle{}baz\textquotesingle{}));

// Nest styles log(chalk.\+red(\textquotesingle{}Hello\textquotesingle{}, chalk.\+underline.\+bg\+Blue(\textquotesingle{}world\textquotesingle{}) + \textquotesingle{}!\textquotesingle{}));

// Nest styles of the same type even (color, underline, background) log(chalk.\+green( \textquotesingle{}I am a green line \textquotesingle{} + chalk.\+blue.\+underline.\+bold(\textquotesingle{}with a blue substring\textquotesingle{}) + \textquotesingle{} that becomes green again!\textquotesingle{} ));

// E\+S2015 template literal log(` C\+PU\+: \$\{chalk.\+red(\textquotesingle{}90\%')\} R\+AM\+: \$\{chalk.\+green(\textquotesingle{}40\textquotesingle{})\} D\+I\+SK\+: \$\{chalk.\+yellow(\textquotesingle{}70\textquotesingle{})\} \`{});

// E\+S2015 tagged template literal log(chalk\`{} C\+PU\+: \{red \$\{cpu.\+total\+Percent\}\%\} R\+AM\+: \{green \$\{ram.\+used / ram.\+total $\ast$ 100\}\%\} D\+I\+SK\+: \{rgb(255,131,0) \$\{disk.\+used / disk.\+total $\ast$ 100\}\%\} \`{});

// Use R\+GB colors in terminal emulators that support it. log(chalk.\+keyword(\textquotesingle{}orange\textquotesingle{})(\textquotesingle{}Yay for orange colored text!\textquotesingle{})); log(chalk.\+rgb(123, 45, 67).underline(\textquotesingle{}Underlined reddish color\textquotesingle{})); log(chalk.\+hex(\textquotesingle{}\#\+D\+E\+A\+D\+ED\textquotesingle{}).bold(\textquotesingle{}Bold gray!\textquotesingle{})); 
\begin{DoxyCode}
Easily define your own themes:

```js
const chalk = require('chalk');

const error = chalk.bold.red;
const warning = chalk.keyword('orange');

console.log(error('Error!'));
console.log(warning('Warning!'));
\end{DoxyCode}


Take advantage of console.\+log \href{https://nodejs.org/docs/latest/api/console.html#console_console_log_data_args}{\tt string substitution}\+:


\begin{DoxyCode}
const name = 'Sindre';
console.log(chalk.green('Hello %s'), name);
//=> 'Hello Sindre'
\end{DoxyCode}


\subsection*{A\+PI}

\subsubsection*{chalk.{\ttfamily $<$style$>$\mbox{[}.$<$style$>$...\mbox{]}(string, \mbox{[}string...\mbox{]})}}

Example\+: `chalk.\+red.\+bold.\+underline(\textquotesingle{}Hello', \textquotesingle{}world\textquotesingle{});\`{}

Chain \href{#styles}{\tt styles} and call the last one as a method with a string argument. Order doesn\textquotesingle{}t matter, and later styles take precedent in case of a conflict. This simply means that {\ttfamily chalk.\+red.\+yellow.\+green} is equivalent to {\ttfamily chalk.\+green}.

Multiple arguments will be separated by space.

\subsubsection*{chalk.\+enabled}

Color support is automatically detected, as is the level (see {\ttfamily chalk.\+level}). However, if you\textquotesingle{}d like to simply enable/disable Chalk, you can do so via the {\ttfamily .enabled} property.

Chalk is enabled by default unless explicitly disabled via the constructor or {\ttfamily chalk.\+level} is {\ttfamily 0}.

If you need to change this in a reusable module, create a new instance\+:


\begin{DoxyCode}
const ctx = new chalk.constructor(\{enabled: false\});
\end{DoxyCode}


\subsubsection*{chalk.\+level}

Color support is automatically detected, but you can override it by setting the {\ttfamily level} property. You should however only do this in your own code as it applies globally to all Chalk consumers.

If you need to change this in a reusable module, create a new instance\+:


\begin{DoxyCode}
const ctx = new chalk.constructor(\{level: 0\});
\end{DoxyCode}


Levels are as follows\+:

0. All colors disabled
\begin{DoxyEnumerate}
\item Basic color support (16 colors)
\item 256 color support
\item Truecolor support (16 million colors)
\end{DoxyEnumerate}

\subsubsection*{chalk.\+supports\+Color}

Detect whether the terminal \href{https://github.com/chalk/supports-color}{\tt supports color}. Used internally and handled for you, but exposed for convenience.

Can be overridden by the user with the flags {\ttfamily -\/-\/color} and {\ttfamily -\/-\/no-\/color}. For situations where using {\ttfamily -\/-\/color} is not possible, add the environment variable {\ttfamily F\+O\+R\+C\+E\+\_\+\+C\+O\+L\+OR=1} to forcefully enable color or {\ttfamily F\+O\+R\+C\+E\+\_\+\+C\+O\+L\+OR=0} to forcefully disable. The use of {\ttfamily F\+O\+R\+C\+E\+\_\+\+C\+O\+L\+OR} overrides all other color support checks.

Explicit 256/\+Truecolor mode can be enabled using the {\ttfamily -\/-\/color=256} and {\ttfamily -\/-\/color=16m} flags, respectively.

\subsection*{Styles}

\subsubsection*{Modifiers}


\begin{DoxyItemize}
\item {\ttfamily reset}
\item {\ttfamily bold}
\item {\ttfamily dim}
\item {\ttfamily italic} $\ast$(Not widely supported)$\ast$
\item {\ttfamily underline}
\item {\ttfamily inverse}
\item {\ttfamily hidden}
\item {\ttfamily strikethrough} $\ast$(Not widely supported)$\ast$
\item {\ttfamily visible} (Text is emitted only if enabled)
\end{DoxyItemize}

\subsubsection*{Colors}


\begin{DoxyItemize}
\item {\ttfamily black}
\item {\ttfamily red}
\item {\ttfamily green}
\item {\ttfamily yellow}
\item {\ttfamily blue} $\ast$(On Windows the bright version is used since normal blue is illegible)$\ast$
\item {\ttfamily magenta}
\item {\ttfamily cyan}
\item {\ttfamily white}
\item {\ttfamily gray} (\char`\"{}bright black\char`\"{})
\item {\ttfamily red\+Bright}
\item {\ttfamily green\+Bright}
\item {\ttfamily yellow\+Bright}
\item {\ttfamily blue\+Bright}
\item {\ttfamily magenta\+Bright}
\item {\ttfamily cyan\+Bright}
\item {\ttfamily white\+Bright}
\end{DoxyItemize}

\subsubsection*{Background colors}


\begin{DoxyItemize}
\item {\ttfamily bg\+Black}
\item {\ttfamily bg\+Red}
\item {\ttfamily bg\+Green}
\item {\ttfamily bg\+Yellow}
\item {\ttfamily bg\+Blue}
\item {\ttfamily bg\+Magenta}
\item {\ttfamily bg\+Cyan}
\item {\ttfamily bg\+White}
\item {\ttfamily bg\+Black\+Bright}
\item {\ttfamily bg\+Red\+Bright}
\item {\ttfamily bg\+Green\+Bright}
\item {\ttfamily bg\+Yellow\+Bright}
\item {\ttfamily bg\+Blue\+Bright}
\item {\ttfamily bg\+Magenta\+Bright}
\item {\ttfamily bg\+Cyan\+Bright}
\item {\ttfamily bg\+White\+Bright}
\end{DoxyItemize}

\subsection*{Tagged template literal}

Chalk can be used as a \href{http://exploringjs.com/es6/ch_template-literals.html#_tagged-template-literals}{\tt tagged template literal}.

\`{}\`{}`js const chalk = require(\textquotesingle{}chalk');

const miles = 18; const calculate\+Feet = miles =$>$ miles $\ast$ 5280;

console.\+log(chalk\`{} There are \{bold 5280 feet\} in a mile. In \{bold \$\{miles\} miles\}, there are \{green.\+bold \$\{calculate\+Feet(miles)\} feet\}. \`{}); 
\begin{DoxyCode}
Blocks are delimited by an opening curly brace (`\{`), a style, some content, and a closing curly brace
       (`\}`).

Template styles are chained exactly like normal Chalk styles. The following two statements are equivalent:

```js
console.log(chalk.bold.rgb(10, 100, 200)('Hello!'));
console.log(chalk`\{bold.rgb(10,100,200) Hello!\}`);
\end{DoxyCode}


Note that function styles ({\ttfamily rgb()}, {\ttfamily hsl()}, {\ttfamily keyword()}, etc.) may not contain spaces between parameters.

All interpolated values ({\ttfamily chalk\`{}\$\{foo\}\`{}}) are converted to strings via the {\ttfamily .to\+String()} method. All curly braces ({\ttfamily \{} and {\ttfamily \}}) in interpolated value strings are escaped.

\subsection*{256 and Truecolor color support}

Chalk supports 256 colors and \href{https://gist.github.com/XVilka/8346728}{\tt Truecolor} (16 million colors) on supported terminal apps.

Colors are downsampled from 16 million R\+GB values to an A\+N\+SI color format that is supported by the terminal emulator (or by specifying {\ttfamily \{level\+: n\}} as a Chalk option). For example, Chalk configured to run at level 1 (basic color support) will downsample an R\+GB value of \#\+F\+F0000 (red) to 31 (A\+N\+SI escape for red).

Examples\+:


\begin{DoxyItemize}
\item `chalk.\+hex('\#\+D\+E\+A\+D\+ED\textquotesingle{}).underline(\textquotesingle{}Hello, world!\textquotesingle{}){\ttfamily  -\/}chalk.\+keyword(\textquotesingle{}orange\textquotesingle{})(\textquotesingle{}Some orange text\textquotesingle{}){\ttfamily  -\/}chalk.\+rgb(15, 100, 204).inverse(\textquotesingle{}Hello!\textquotesingle{})\`{}
\end{DoxyItemize}

Background versions of these models are prefixed with {\ttfamily bg} and the first level of the module capitalized (e.\+g. {\ttfamily keyword} for foreground colors and {\ttfamily bg\+Keyword} for background colors).


\begin{DoxyItemize}
\item `chalk.\+bg\+Hex('\#\+D\+E\+A\+D\+ED\textquotesingle{}).underline(\textquotesingle{}Hello, world!\textquotesingle{}){\ttfamily  -\/}chalk.\+bg\+Keyword(\textquotesingle{}orange\textquotesingle{})(\textquotesingle{}Some orange text\textquotesingle{}){\ttfamily  -\/}chalk.\+bg\+Rgb(15, 100, 204).inverse(\textquotesingle{}Hello!\textquotesingle{})\`{}
\end{DoxyItemize}

The following color models can be used\+:


\begin{DoxyItemize}
\item \href{https://en.wikipedia.org/wiki/RGB_color_model}{\tt {\ttfamily rgb}} -\/ Example\+: `chalk.\+rgb(255, 136, 0).bold(\textquotesingle{}Orange!'){\ttfamily }
\item {\ttfamily \mbox{[}}hex{\ttfamily \mbox{]}(\href{https://en.wikipedia.org/wiki/Web_colors#Hex_triplet}{\tt https\+://en.\+wikipedia.\+org/wiki/\+Web\+\_\+colors\#\+Hex\+\_\+triplet}) -\/ Example\+:}chalk.\+hex(\textquotesingle{}\#\+F\+F8800\textquotesingle{}).bold(\textquotesingle{}Orange!\textquotesingle{}){\ttfamily }
\item {\ttfamily \mbox{[}}keyword{\ttfamily \mbox{]}(\href{https://www.w3.org/wiki/CSS/Properties/color/keywords}{\tt https\+://www.\+w3.\+org/wiki/\+C\+S\+S/\+Properties/color/keywords}) (C\+SS keywords) -\/ Example\+:}chalk.\+keyword(\textquotesingle{}orange\textquotesingle{}).bold(\textquotesingle{}Orange!\textquotesingle{}){\ttfamily }
\item {\ttfamily \mbox{[}}hsl{\ttfamily \mbox{]}(\href{https://en.wikipedia.org/wiki/HSL_and_HSV}{\tt https\+://en.\+wikipedia.\+org/wiki/\+H\+S\+L\+\_\+and\+\_\+\+H\+SV}) -\/ Example\+:}chalk.\+hsl(32, 100, 50).bold(\textquotesingle{}Orange!\textquotesingle{}){\ttfamily }
\item {\ttfamily \mbox{[}}hsv{\ttfamily \mbox{]}(\href{https://en.wikipedia.org/wiki/HSL_and_HSV}{\tt https\+://en.\+wikipedia.\+org/wiki/\+H\+S\+L\+\_\+and\+\_\+\+H\+SV}) -\/ Example\+:}chalk.\+hsv(32, 100, 100).bold(\textquotesingle{}Orange!\textquotesingle{}){\ttfamily }
\item {\ttfamily \mbox{[}}hwb{\ttfamily \mbox{]}(\href{https://en.wikipedia.org/wiki/HWB_color_model}{\tt https\+://en.\+wikipedia.\+org/wiki/\+H\+W\+B\+\_\+color\+\_\+model}) -\/ Example\+:}chalk.\+hwb(32, 0, 50).bold(\textquotesingle{}Orange!\textquotesingle{}){\ttfamily  -\/}ansi16{\ttfamily  -\/}ansi256\`{}
\end{DoxyItemize}

\subsection*{Windows}

If you\textquotesingle{}re on Windows, do yourself a favor and use \href{http://cmder.net/}{\tt {\ttfamily cmder}} instead of {\ttfamily cmd.\+exe}.

\subsection*{Origin story}

\href{https://github.com/Marak/colors.js}{\tt colors.\+js} used to be the most popular string styling module, but it has serious deficiencies like extending {\ttfamily String.\+prototype} which causes all kinds of \href{https://github.com/yeoman/yo/issues/68}{\tt problems} and the package is unmaintained. Although there are other packages, they either do too much or not enough. Chalk is a clean and focused alternative.

\subsection*{Related}


\begin{DoxyItemize}
\item \href{https://github.com/chalk/chalk-cli}{\tt chalk-\/cli} -\/ C\+LI for this module
\item \href{https://github.com/chalk/ansi-styles}{\tt ansi-\/styles} -\/ A\+N\+SI escape codes for styling strings in the terminal
\item \href{https://github.com/chalk/supports-color}{\tt supports-\/color} -\/ Detect whether a terminal supports color
\item \href{https://github.com/chalk/strip-ansi}{\tt strip-\/ansi} -\/ Strip A\+N\+SI escape codes
\item \href{https://github.com/chalk/strip-ansi-stream}{\tt strip-\/ansi-\/stream} -\/ Strip A\+N\+SI escape codes from a stream
\item \href{https://github.com/chalk/has-ansi}{\tt has-\/ansi} -\/ Check if a string has A\+N\+SI escape codes
\item \href{https://github.com/chalk/ansi-regex}{\tt ansi-\/regex} -\/ Regular expression for matching A\+N\+SI escape codes
\item \href{https://github.com/chalk/wrap-ansi}{\tt wrap-\/ansi} -\/ Wordwrap a string with A\+N\+SI escape codes
\item \href{https://github.com/chalk/slice-ansi}{\tt slice-\/ansi} -\/ Slice a string with A\+N\+SI escape codes
\item \href{https://github.com/qix-/color-convert}{\tt color-\/convert} -\/ Converts colors between different models
\item \href{https://github.com/bokub/chalk-animation}{\tt chalk-\/animation} -\/ Animate strings in the terminal
\item \href{https://github.com/bokub/gradient-string}{\tt gradient-\/string} -\/ Apply color gradients to strings
\item \href{https://github.com/LitoMore/chalk-pipe}{\tt chalk-\/pipe} -\/ Create chalk style schemes with simpler style strings
\item \href{https://github.com/sindresorhus/terminal-link}{\tt terminal-\/link} -\/ Create clickable links in the terminal
\end{DoxyItemize}

\subsection*{Maintainers}


\begin{DoxyItemize}
\item \href{https://github.com/sindresorhus}{\tt Sindre Sorhus}
\item \href{https://github.com/qix-}{\tt Josh Junon}
\end{DoxyItemize}

\subsection*{License}

M\+IT 